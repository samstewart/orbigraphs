% EJC papers *must* begin with the following two lines. 
\documentclass[12pt]{article}
\usepackage{e-jc}

% Please remove all other commands that change parameters such as
% margins or pagesizes.

% only use standard LaTeX packages
% only include packages that you actually need

% we recommend these ams packages
\usepackage{amsthm,amsmath,amssymb}

% we recommend the graphicx package for importing figures
\usepackage{graphicx}

% use this command to create hyperlinks (optional and recommended)
\usepackage[colorlinks=true,citecolor=black,linkcolor=black,urlcolor=blue]{hyperref}

% use these commands for typesetting doi and arXiv references in the bibliography
\newcommand{\doi}[1]{\href{http://dx.doi.org/#1}{\texttt{doi:#1}}}
\newcommand{\arxiv}[1]{\href{http://arxiv.org/abs/#1}{\texttt{arXiv:#1}}}

% all overfull boxes must be fixed; 
% i.e. there must be no text protruding into the margins


% declare theorem-like environments
\theoremstyle{plain}
\newtheorem{theorem}{Theorem}
\newtheorem{lemma}[theorem]{Lemma}
\newtheorem{corollary}[theorem]{Corollary}
\newtheorem{proposition}[theorem]{Proposition}
\newtheorem{fact}[theorem]{Fact}
\newtheorem{observation}[theorem]{Observation}
\newtheorem{claim}[theorem]{Claim}

\theoremstyle{definition}
\newtheorem{definition}[theorem]{Definition}
\newtheorem{example}[theorem]{Example}
\newtheorem{conjecture}[theorem]{Conjecture}
\newtheorem{open}[theorem]{Open Problem}
\newtheorem{problem}[theorem]{Problem}
\newtheorem{question}[theorem]{Question}

\theoremstyle{remark}
\newtheorem{remark}[theorem]{Remark}
\newtheorem{note}[theorem]{Note}

%%%%%%%%%%%%%%%%%%%%%%%%%%%%%%%%%%%%%%%%%%%%%%%%%%%%%%%

% if needed include a line break (\\) at an appropriate place in the title

\title{\bf Orbigraphs: Defining a Graph Theoretic Analog to Reimannian Orbifolds}

% input author, affilliation, address and support information as follows;
% the address should include the country, and does not have to include
% the street address 

\author{Dagwood Remifa\thanks{Supported by NASA grant ABC123.}\\
\small Department of Inconsequential Studies\\[-0.8ex]
\small Solatido College\\[-0.8ex] 
\small North Kentucky, U.S.A.\\
\small\tt remifa@dis.solatido.edu\\
\and
Forgotten Second Author \qquad  Forgotten Third Author\\
\small School of Hard Knocks\\[-0.8ex]
\small University of Western Nowhere\\[-0.8ex]
\small Nowhere, Australasiaopia\\
\small\tt \{fsa,fta\}@uwn.edu.ao
}

% \date{\dateline{submission date}{acceptance date}\\
% \small Mathematics Subject Classifications: comma separated list of
% MSC codes available from http://www.ams.org/mathscinet/freeTools.html}

\date{\dateline{Jan 1, 2012}{Jan 2, 2012}\\
\small Mathematics Subject Classifications: 05C88, 05C89}

\begin{document}

\maketitle

% E-JC papers must include an abstract. The abstract should consist of a
% succinct statement of background followed by a listing of the
% principal new results that are to be found in the paper. The abstract
% should be informative, clear, and as complete as possible. Phrases
% like "we investigate..." or "we study..." should be kept to a minimum
% in favor of "we prove that..."  or "we show that...".  Do not
% include equation numbers, unexpanded citations (such as "[23]"), or
% any other references to things in the paper that are not defined in
% the abstract. The abstract will be distributed without the rest of the
% paper so it must be entirely self-contained.

\begin{abstract}
  % keywords are optional
  \bigskip\noindent \textbf{Keywords:} orbifolds, spectral graph theory
\end{abstract}

\section{Introduction}

Introduction here.

%%%%%%%%%%%%%%%%%%%%%%%%%%%%%%%%%%%%%%%%%%%%%%%%%%%%%%%
\section{Brief introduction to Riemannian orbifolds}

Brooks analogy, Bass Serre Theory mention.

%%%%%%%%%%%%%%%%%%%%%%%%%%%%%%%%%%%%%%%%%%%%%%%%%%%%%%%
\section{Defining Orbigraphs as Analogs of Orbifolds}

Reasons for choices, etc.
\begin{definition}
A $k$-\textbf{orbigraph} is...
\end{definition}

%%%%%%%%%%%%%%%%%%%%%%%%%%%%%%%%%%%%%%%%%%%%%%%%%%%%%%%
\section{Properties of Orbigraphs}

\subsection{Covering Theory for Orbigraphs}

\begin{definition}
An \textbf{equitable partition} of an orbigraph $\Gamma$ is...
\end{definition}

We say that an orbigraph $\Gamma_1$ covers an orbigraph $\Gamma_2$ if there is an equitable partition $\pi$ of $\Gamma$ such that $\Gamma_1 / \pi = \Gamma_2$.

\begin{lemma}
The covering relation is transitive and reflexive. (Orbigraphs form a category.)
\end{lemma}

\begin{theorem}
Every connected $k$-orbigraph is covered by the $k$-tree. (Disconnected orbigraphs are covered by disjoint unions of $k$-trees.)
\end{theorem}

These two previous results imply that we can think of $k$-trees as universal covers of orbigraphs.

We are particularly interested in the case when an orbigraph $\Gamma$ is covered by a finite $k$-regular graph. We call such orbigraphs \textbf{good} which parallels the corresponding distinction in orbifold theory. We have characterized when an orbigraph is good:

\begin{theorem}
An orbigraph is good if and only if, for every sequence of vertices $v_1,\ldots,v_n$ such that $v_1 = v_n$ the number of walks forward is the same as the number backward.
\end{theorem}

\subsection{Group Actions on Orbigraphs}

We can also investigate the relationship between group actions and orbigraphs:

\begin{theorem}
If a group $G$ acts on $\Gamma$ by automorphisms, then $\Gamma / G$ is an orbigraph and $\Gamma$ covers $\Gamma / G$.
\end{theorem}

Note that this implies that an orbigraph is good if it is the quotient of a $k$-regular graph by a group action. We see that this group action can be lifted to the universal covering tree:

\begin{theorem}
If $\Gamma$ is a $k$-orbigraph such that $\Gamma = \widetilde{\Gamma}/G$ for some automorphism group $G$, then there exists a group of automorphism $\widetilde{G}$ of $T_k$ such that $T_k / \widetilde{G} = \Gamma$.
\end{theorem}

Such a group may be regarded as the fundamental group of the orbigraph.

%%%%%%%%%%%%%%%%%%%%%%%%%%%%%%%%%%%%%%%%%%%%%%%%%%%%%%%
\section{Spectral properties of orbigraphs}

There are well developed spectral theories of manifolds and orbifolds. Showing that spectral results about orbigraphs parallel these theories is a good way to establish their legitimacy.

\subsection{Familiar Spectral Results}

Many spectral results about manifolds carry over to orbifolds. The same is true for regular graphs vs orbigraphs. We investigate the spectrum of the adjacency operator of an orbigraph.

The following lemma is of fundamental importance for the spectral theory of orbigraphs:

\begin{lemma}
If $\widetilde{\Gamma}$ covers $\Gamma$, then $Spec(\Gamma) \subseteq Spec(\widetilde{\Gamma})$ as multisets, with equality only when the covering is discrete.
\end{lemma}

\begin{theorem}
The spectral radius of a $k$-orbigraph is $k$.
\end{theorem}

\begin{theorem}
The spectrum of an orbigraph determines and is determined by length spectrum.
\end{theorem}

\begin{theorem}
An orbigraph is bipartite if and only if its spectrum is symmetric about zero.
\end{theorem}


\subsection{Results Intrinsic to Orbigraphs}

\begin{theorem}
Orbigraphs with complex eigenvalues are bad.
\end{theorem}

Note that complex eigenvalues mean that there is an imbalanced loop.

As in the orbifold case (reference or quote theorem) we can establish bounds on the singular set of an orbigraph via the spectrum:

\begin{theorem}
If $\Gamma$ is a $k$-orbigraph on $n$-vertices with $s$ singular points then we have... (bound on $s$).
\end{theorem}

%%%%%%%%%%%%%%%%%%%%%%%%%%%%%%%%%%%%%%%%%%%%%%%%%%%%%%%
\subsection{Cospectral Orbigraphs}

Here we show that some familiar methods for generating cospectral graphs can be extended to the orbifold case. In particular we extend Sunada's theorem to orbifolds.

\begin{theorem}
If $\Gamma$ is an orbigraph and $H_1$ and $H_2$ are Sunada equivalent subgroups of the automorphism group of $\Gamma$, then $spec(\Gamma / H_1) = spec(\Gamma / H_2)$.
\end{theorem}

Example using $K_6$ with gadgets.

The method of Seidel switching (reference) can also be extended to orbigraphs:

\begin{theorem}
Complicated statement about cospectral covers.
\end{theorem}

This gives a simple method of finding cospectral good and bad orbigraphs:

\begin{theorem}
There are cospectral good and bad orbigraphs.
\end{theorem}


%%%%%%%%%%%%%%%%%%%%%%%%%%%%%%%%%%%%%%%%%%%%%%%%%%%%%%%
\section{Conclusion}
Some concluding remarks.


%%%%%%%%%%%%%%%%%%%%%%%%%%%%%%%%%%%%%%%%%%%%%%%%%%%%%%%
\subsection*{Acknowledgements}
Acknowledgements here.

%%%%%%%%%%%%%%%%%%%%%%%%%%%%%%%%%%%%%%%%%%%%%%%%%%%%%%%
% \bibliographystyle{plain} 
% \bibliography{myBibFile} 
% If you use BibTeX to create a bibliography
% then copy and past the contents of your .bbl file into your .tex file

\begin{thebibliography}{10}

\bibitem{Bollobas} B{\'e}la Bollob{\'a}s.  \newblock Almost every
  graph has reconstruction number three.  \newblock {\em J. Graph
    Theory}, 14(1):1--4, 1990.

\bibitem{WikipediaReconstruction} Wikipedia contributors.  \newblock
  Reconstruction conjecture.  \newblock {\em Wikipedia, the free
    encyclopedia}, 2011.

\bibitem{FGH} J.~Fisher, R.~L. Graham, and F.~Harary.  \newblock A
  simpler counterexample to the reconstruction conjecture for
  denumerable graphs.  \newblock {\em J. Combinatorial Theory Ser. B},
  12:203--204, 1972.

\bibitem{HHRT} Edith Hemaspaandra, Lane~A. Hemaspaandra,
  Stanis{\l}aw~P. Radziszowski, and Rahul Tripathi.  \newblock
  Complexity results in graph reconstruction.  \newblock {\em Discrete
    Appl. Math.}, 155(2):103--118, 2007.

\bibitem{Kelly} Paul~J. Kelly.  \newblock A congruence theorem for
  trees.  \newblock {\em Pacific J. Math.}, 7:961--968, 1957.

\bibitem{KSU} Masashi Kiyomi, Toshiki Saitoh, and Ryuhei Uehara.
  \newblock Reconstruction of interval graphs.  \newblock In {\em
    Computing and combinatorics}, volume 5609 of {\em Lecture Notes in
    Comput. Sci.}, pages 106--115. Springer, 2009.

\bibitem{RM} S.~Ramachandran and S.~Monikandan.  \newblock Graph
  reconstruction conjecture: reductions using complement, connectivity
  and distance.  \newblock {\em Bull. Inst. Combin. Appl.},
  56:103--108, 2009.

\bibitem{RR} David Rivshin and Stanis{\l}aw~P. Radziszowski.
  \newblock The vertex and edge graph reconstruction numbers of small
  graphs.  \newblock {\em Australas. J. Combin.}, 45:175--188, 2009.

\bibitem{Stockmeyer} Paul~K. Stockmeyer.  \newblock The falsity of the
  reconstruction conjecture for tournaments.  \newblock {\em J. Graph
    Theory}, 1(1):19--25, 1977.

\bibitem{Ulam} S.~M. Ulam.  \newblock {\em A collection of
    mathematical problems}.  \newblock Interscience Tracts in Pure and
  Applied Mathematics, no. 8.  Interscience Publishers, New
  York-London, 1960.

\end{thebibliography}

\end{document}
For the first direction of Theorem \ref{thm:GoodBadCharacterization}, we wish to show that if an orbigraph $\Gamma$, when converted to a Markov-chain is reversible, then $\Gamma$ is good (has a finite $k$-regular cover). Because $\Gamma$ has constant out-degree $k$, we can easily convert $\Gamma$ to a Markov chain by normalizing the rows of its adjacency matrix $A$ and thus treat $\Gamma$ as a Markov chain henceforward.

Reversible Markov chains satisfy the \textit{detailed balance} equation 

\begin{equation}\label{eq:DetailedBalance}
	\pi_i A_{ i, j } = \pi_j A_{ j, i }
\end{equation}

where $\pi$ is the stationary distribution of $\Gamma$. By reversibility, $\pi$ is also rational [Kemney and Snell] so that \ref{eq:DetailedBalance} becomes

\begin{equation} \label{eq:DetailedBalanceNormalized}
	d_i A_{i, j} = d_j A_{ i, j }
\end{equation}

where $d_i$ and $d_j$ are integers.

We use the detailed balance equation to construct the finite $k$-regular cover $C$ with partition classes corresponding to the vertices of $\Gamma$. Every vertex in class $i$ of $C$ must connect to $A_{i, j}$ vertices in class $j$ and, conversely, every vertex in class $j$ must connect to $A_{j, i}$ vertices in parition $i$.  Equivalently, $A_{j, i}$ should divide the number of vertices in class $i$ and $A_{i, j}$ should divide the number of vertices in class $j$. Thus, we choose $c$ such that $A_{j, i} | c d_i$ and $A_{i, j} | c d_j$ so that \ref{eq:DetailedBalanceNormalized} becomes

\begin{equation} \label{eq:DetailedBalanceWithConstant}
	(c d_i) A_{ i, j } = (c d_j) A_{ i, j }
\end{equation}

and we place $c d_i$ vertices in class $i$ and $c d_j$ vertices in class $j$.

Connections \textit{within} classes require a special treatment. Every vertex in class $i$ must connect to $A_{i, i}$ other vertices its own class, so in total, we need a multiple of $A_{i, i} + 1$ vertices in class $i$. Thus, we choose $c$ such that $(A_{ i, i } + 1) | c d_i$. 

This construction produces a $k$-regular cover with an equitable partition that quotients to $\Gamma$. As outlined above, our choice of $c$ ensures the proper number of vertices and edges and \ref{eq:DetailedBalanceWithConstant} guarantees an equitable partition.

For the other direction of \ref{thm:GoodBadCharacterization}, we rely on [Boyd] who proves that an undirected covering graph $C$ quotients to a reversible graph $\Gamma$. 
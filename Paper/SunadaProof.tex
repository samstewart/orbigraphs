In this proof $H_i$ will refer to either $H_1$ or $H_2$. Also, let $E_\lambda^{\Gamma}$ represent the $\lambda$-eigenspace of $A_\Gamma$. By Lemma \ref{lemma:SpectrumInclusion}, we have that $ \spec(\Gamma / H_i) \subseteq \spec(\Gamma) $. Then to prove the theorem, we must show that the $\operatorname{dim} E_\lambda^{H_i}$ is independent of $i$. (Note that if $\lambda$ is not an eigenvalue of $A_{\Gamma_i}$ then we may say that the corresponding $\lambda$-eigenspace has dimension zero.)

Consider the representation $\rho_G$ of $G$ on $\mathbb{C}^{V(\Gamma)}$ given for $\phi \in \mathbb{C}^{V(\Gamma)}$ and $v \in V(\Gamma)$ by:
$$
[\rho_G(g)\phi](v) = \phi(g^{-1}.v).
$$
Note that this is a valid representation. First let $z,w \in \mathbb{C}$ and $\phi, \psi \in \mathbb{C}^{V(\Gamma)}$. Then we have
\begin{align*}
    [ \rho_G(g)(z \phi + w \psi) ](v) &= (z \phi + w \psi)(g^{-1}.v) \\
    &= z \cdot \phi(g^{-1}.v) + w \cdot \psi(g^{-1}.v) \\
    &= z \cdot [ \rho_G(g)\phi ](v) + z \cdot [ \rho_G(g)\psi ](v).
\end{align*}
Thus, $\rho_G$ is linear. Next, let $g, h \in G$ and consider:
\begin{align*}
    [ \rho_G(gh)\phi ](v) &= \phi((gh)^{-1}.v) \\
    &= \phi(h^{-1}g^{-1}.v).
\end{align*}
Whereas:
\begin{align*}
    [ (\rho_G(g) \circ \rho_G(h))\phi ](v) &= (\rho_G(h)\phi)(g^{-1}.v) \\
    &= \phi(h^{-1}g^{-1}.v).
\end{align*}
Thus, $\rho_G$ is a homeomorphism. Now we will show that a $\lambda$-eigenspace of $A_\Gamma$ is invariant under the action of $\rho_G$. Let $E_{\lambda}^{\Gamma}$ be such an eigenspace and $\phi \in E_{\lambda}^{\Gamma}$. We would like to show that $A_\Gamma (\rho_G(g)\phi) = \lambda (\rho_G(g)\phi)$. We have the following:
\begin{align*}
[A_\Gamma (\rho_G(g)\phi)](v) &= \sum_{u \in \Gamma} w(v, u)[ \rho_G(g)\phi ](u) \\
&= \sum_{u \in \Gamma} w(v, u)\phi(g^{-1}.u) \\
&= \sum_{u \in \Gamma} w(g^{-1}.v, g^{-1}.u)\phi(g^{-1}.u)
\end{align*}

where the last equality is because $G$ acts by automorphisms. Of course, this is simply $[A_\Gamma \phi](g^{-1}.v) = \lambda \phi(g^{-1}.v)$. On the other hand, we have $\lambda [\rho_G(g)\phi](v) = \lambda \phi(g^{-1}.v)$, giving the desired result.

Because $E_{\lambda}^{\Gamma}$ is invariant under $\rho_G$, there is a subrepresentation $\rho_G^\lambda : G \to \operatorname{GL}(E_{\lambda}^{\Gamma})$ which has degree equal to the dimension $E_{\lambda}^{\Gamma}$. 

Now consider $E_{\lambda}^{\Gamma_i}$, the $\lambda$-eigenspace of $A_{\Gamma_i}$. I will show that its dimension is equal to the dimension of the subspace of $E_{\lambda}^{\Gamma}$ that is left fixed by the action of $\operatorname{Res}_{H_i}^G \rho_G^\lambda$, the restriction of $\rho_G^\lambda$ to $H_i$. Let $\varphi_1 \ldots \varphi_m$ be a basis for $E_{\lambda}^{\Gamma_i}$. First, note that by Lemma \ref{lemma:SpectrumInclusion}, each of the $\varphi_i$ lifts to a $\lambda$-eigenfunction of $A_\Gamma$. Also, the lifts of the $\varphi_i$ remain linearly independent and are clearly fixed by the action of $\operatorname{Res}_{H_i}^G \rho_G^\lambda$. Furthermore, any element of $E_{\lambda}^{\Gamma}$ that is fixed by $\operatorname{Res}_{H_i}^G \rho_G^\lambda$ must be the lift of some element of $E_{\lambda}^{\Gamma_i}$. Therefore the $\varphi_i$ span the fixed subspace and the dimensions are equal, as claimed.

By elementary represenation theory we have that the dimension of the subspace of $E_{\lambda}^{\Gamma}$ fixed by $\operatorname{Res}_{H_i}^G \rho_G^\lambda$ is equal to $\langle 1_{H_i}, \operatorname{Res}_{H_i}^G \rho_G^\lambda \rangle$. However, by Frobenius reciprocity we have that 
$$
    \langle 1_{H_i}, \operatorname{Res}_{H_i}^G \rho_G^\lambda \rangle = \langle \ind_{H_i}^G 1_{H_i}, \rho_G^\lambda \rangle.
$$
By Lemma \ref{lemma:SunadaEquivRepresenation} $\ind_{H_1}^G 1_{H_1}$ is equivalent to $\ind_{H_2}^G 1_{H_2}$, therefore
$$
    \langle \ind_{H_1}^G 1_{H_1}, \rho_G^\lambda \rangle = \langle \ind_{H_2}^G 1_{H_2}, \rho_G^\lambda \rangle
$$
for all $\lambda$. Therefore, the dimensions of $E_{\lambda}^{\Gamma_1}$ and $E_{\lambda}^{\Gamma_2}$ are equal for all $\lambda$ and so $\Gamma_1$ and $\Gamma_2$ are cospectral.
